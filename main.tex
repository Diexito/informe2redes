\documentclass[spanish]{udpreport}
\usepackage[utf8]{inputenc}
\usepackage[spanish]{babel}
\usepackage{textcomp}
\usepackage{graphicx}
\graphicspath{ {images/} }

% Podemos establecer el logo de alguna entidad o dejar el de la UDP (defecto)
%\setlogo{EITFI}


\begin{document}
\title{
	Creación de paquetes utilizando Scapy y validación con Wireshark\\[2ex]
	\normalsize
	Laboratorio N°3, Redes de Datos\\
    Profesor: Jaime Álvarez\\
	Ayudante: Maximiliano Vega
    }
\author{John Bidwell Boitano\\ Diego Aguilera Morán \\ Valentín Morales Albornoz \\ Yerson Sarria Quiroz}
\email{john.bidwell@mail.udp.cl\\diego.aguileramo@mail.udp.cl\\valentin.morales@mail.udp.cl\\yerson.sarria@mail.udp.cl}
\date{14 de abril de 2016}
\maketitle

\tableofcontents

\chapter{Introducción}
En este laboratorio analisaremos el flujo de paquetes en una red, diferenciando entre los paquetes que fluyen en esta.
Ademas de analisar estos paquetes podremos crear paquetes personalizados usando el software Scapy, pero antes de esto debemos saber como desglosar un paquete por capas y entender el funcionamiento de cada capa.
\chapter{Cuestionario}
\begin{enumerate}

\item¿Qué  pasa  cuando  envió  un  paquete  a  la  dirección  FF:FF:FF:FF:FF:FF?  ¿Quienes 
lo reciben? ¿Por qué?

\item¿Qué  pasa  cuando  envió  un  paquete  a  una  MAC  de  otro  equipo?  ¿Quieres  lo pueden reciben? ¿Por qué?

\item¿Qué  sucede  si  envía  un  paquete  a  una  MAC  que  no  corresponda  a  ningún  equipo de la red? ¿Quienes lo pueden recepcionar? ¿Por qué?

\end{enumerate}

\end{document}